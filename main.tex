\documentclass{article}
\usepackage[utf8]{inputenc}
\usepackage[margin=1.5in]{geometry}
\usepackage{booktabs}
\usepackage[labelfont=bf, skip=5pt, font =small]{caption}
\usepackage{subcaption}
\usepackage{tabu}
\usepackage[style=chem-acs, articletitle = true]{biblatex}
\usepackage[siunitx]{circuitikz}
\usepackage{amsmath, amssymb,amsfonts}

\addbibresource{references.bib}

\usepackage{graphicx}
\usepackage{fancyhdr}

\setlength{\parskip}{1em}
\setlength{\parindent}{0em}


\begin{document}

\begin{figure}[!t]
    \centering
     \includegraphics[width=\textwidth]{polito_logo_2021_blu.jpeg}
     \label{fig:logo_polito}
\end{figure}
\begin{center}
    \huge{Laboratorio 1: Oscilloscopio Digitale}\\[10pt]
    \large{Relazione a cura di:
    \\Viorel Strogoteanu\\Luigi Tatonetti\\Michele Pettiti\\Giovanni Lombardi}
\end{center}


\rule{\textwidth}{0.5pt}


    \noindent Obiettivi del laboratorio: \\
\rule{\textwidth}{0.5pt}

\section{Valore efficace e frequenza}
    \subsection{Operazioni preliminari}

\subsubsection{Cenni teorici}
                \begin{circuitikz}\draw
(0,0) to[european voltage source, l_=$V_g$] (0,4)
      to[R, l_=$R_g$](4,4)
      to[C, l_=$C_s$](5,4) -- (6,4)
      to[C, l_=$C_c$](6,1) --(6,0)
(6,0) -- (0,0)
    
;
\end{circuitikz}
\subsection{Misurazione valor efficace}
   $V_{pp} = 1.000 $ V
   
   

$\dfrac{\delta{\text{$V_{pp}$}}}{\text{$V_{pp}$}}$=$
$\dfrac{\delta{\text{$nDIV$}}}{\text{$nDIV$}}$ + \dfrac{\delta{\text{$S_v$}}}{\text{$S_v$}}$ = 4\% + 3\% = \pm 7\%
 
$\delta{\text{$V_{pp}$}} = 0.07 $ V

$V_{eff} $= $$\dfrac{\delta{\text{$v_{max}$}}}{\text{$\sqrt{2}$}$

    


\subsection{Misurazione di frequenza}
\subsection{Verifica con multimetro}
    
\section{Misurazione del tempo di salita}
\subsection{Operazioni preliminari}
\subsection{Misurazione 1: tempo di salita in condizioni di adattamento di impedenza}
\subsection{Misurazione 2: tempo di salita con generatore di alta impedenza(uso della sonda compensata)}

\section{Aliasing}
\subsection{Operazioni preliminari e visualizzazione in assenza di aliasing}
\subsection{Aliasing percettivo}
\subsection{Effetto dell'aliasing nel dominio del tempo}
    
\section{Conclusioni}
    
    
    
\end{document}
